%!TEX root = stoh_modeling.tex
\section*{Введение}
\addcontentsline{toc}{section}{Введение}
В наше время биоинформатика является неотъемлимоий частью практически всех областеий биологии. Наиболее значим вклад биоинформатики в области генетики. С помощью введения методов биоинформатики гораздо упростилась работа с большим количеством информации о структуре и последовательности генетического материала.

Одним из инструментов исследования ДНК стало использование ферментов рестрикции (рестрикционное картирование). Они позволяют превращать молекулы ДНК очень большого размера в набор фрагментов длиной от нескольких сотен до нескольких тысяч оснований.

С помощью рестрикционного картирования генома можно идентифицировать на ДНК биологически важные участки. Метод рестрикционного картирования позволяет выделить в геноме крупные генетические изменения, такие как хромосомные перестроийки (делеция или инсерция). Метод использования рестрикционных карт рассматривается как скоринговыий метод анализа стуктурных вариациий и в целом для анализа регионов генома не поддающихся сборке с помощью методов основанных на нуклиотидных последовательностях, так как значительно менее трудозатратен и менее информативен, поэтому он менее изучен.
Природа рестрикционных карт отличается от такавой для нуклиотидных последовательностей, поэтому необходимо учитывать особенность модели при рассмотрении методов их анализа.

В рамках проблемы карты генома WGM Assembly \cite{wgm}, основанного на множественном выравнивании рестрикционных карт, время попарного выравнивания составляет существенную часть общего времени работы алгоритма \cite{wgmassembly}. Поэтому особое значение приобретает задача его ускорения. В работе рассматриваются модификации метода парного выравнивания, предложенного в статьях Смита-Ватермана \cite{smith_water}, за счет использования информации об относительном смещении карт друг относительно друга, позволяющие существенно ускорить время его работы.
