%!TEX root = stoh_modeling.tex
\section{Постановка задачи}
Время работы алгоритма попарного выравнивания, основанного на динамическом заполнении матрицы выравниваний можно оценить как $O(n^2)$, где $n$ - максимально возможное количество разрезов на карте. Целью данной работы является описание и исследование двух эвристических методов, которые позволяют его ускорить, и добиться оценки среднего времени работы $O(n)$ для каждого из них. В данной работе будет проведено сравнение двух этих методов между собой и с исходным алгоритмом. Будет показано, что использование этих методов предпочтительно в случае определенной модели структуры множества исходных рестрикционных карт, а так же будут оценены границы применимости данных эвристик.
