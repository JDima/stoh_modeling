%!TEX root = stoh_modeling.tex
\section{Инструменты для моделирования}
Были рассмотрены следующие пакеты для стохастического моделирования - fern\cite{fern}, StochKit\cite{stochkit}, StochPy\cite{stochpy},
GillesPy\cite{stochkit}.
Каждый из них рассчитан на языки программирования - java, c++, python соотвественно. Далее будет каждый пакет рассмотрен более подробно,
но наш выбор остановился на StochKit, как на самом быстром.
\subsection{StochPy, GillesPy}
Является обвёрткой над пакетом StochKit с некоторыми модификациями. В комплекте с моделирование есть возможность строить графики.
Содержит две модификации  алгоритма гиллеспи. На вход не принимает файлы в формате sbml (нужно конвертировать в xml).
\subsection{fern}
Является самым медленным из предложенных пакетов стохастического моделирования. Содержит более семи модификаций алгоритма гиллеспи. На вход
принимает файлы в формате sbml, что является большим плюсом. Данный пакет подходит для моделирования небольшого количества реакций.
\subsection{StochKit}
Является самым быстрым, так как написан на C и все остальные пакеты либо опираются на него, либо являются обвёрткой. В комплекте идёт
программа, написанная на Matlab, для построения графиков, но к сожалению данные графики на неинтересны. В пакете реализовано несколько
модификаций алгоритма гиллеспи. Выходом программы StochKit является текстовый файл с концентрациями веществ в любой момент времени (частота
задаётся параметром).
\subsection{DEEP}
Наша модель будет содержать параметры и для моделирования мы воспользуемся генетическими алгоритмами, реализованными в DEEP \cite{deep}.
